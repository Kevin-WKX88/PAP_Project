\documentclass[a4paper, 12pt]{article}

\usepackage[utf8]{inputenc} % POUR MODIFIER : ALLER DANS PREFERENCE TEXSHOP
\usepackage[T1]{fontenc}
\usepackage[francais]{babel}
\usepackage{verbatim}
\usepackage{moreverb}
\usepackage{algorithm, algpseudocode}%Meilleur que listing???

\usepackage{listings} % Pour écrire algo

\usepackage{listingsutf8}

%Paramètrage avec : 
\renewcommand{\lstlistingname}{Interface}
%\lstset{
%basicstyle=\footnotesize,    % taille de la police du code
%numbers=left,   % placer le numéro de chaque ligne à gauche (left) 
%numberstyle=\normalsize,    % taille de la police des numéros
%numbersep=8pt,  % distance entre le code et sa numérotation
%}
\lstset{
language=c++,        % choix du langage de programmation 
  frame=single,
  basicstyle=\small\normalfont\sffamily,    % the size of the fonts that are used for the code
  stepnumber=1,                           % the step between two line-numbers. If it is 1 each line will be numbered
 % numbersep=10pt,                         % how far the line-numbers are from the code
  tabsize=1,                              % tab size in blank spaces
  breaklines=true,                        % sets automatic line breaking
  captionpos=t,                           % sets the caption-position to top
  mathescape=true,
  %stringstyle=\color{white}\ttfamily, % Farbe der String
  showspaces=false,           % Leerzeichen anzeigen ?
  showtabs=false,             % Tabs anzeigen ?
  xleftmargin=17pt,
  framexleftmargin=17pt,
  framexrightmargin=17pt,
  framexbottommargin=5pt,
  framextopmargin=5pt,
  showstringspaces=false,
  inputencoding=utf8,
  extendedchars=true,
  literate=%
            {é}{{\'{e}}}1
            {è}{{\`{e}}}1
            {ê}{{\^{e}}}1
            {ë}{{\¨{e}}}1
            {û}{{\^{u}}}1
            {ù}{{\`{u}}}1
            {â}{{\^{a}}}1
            {à}{{\`{a}}}1
            {î}{{\^{i}}}1
            {ô}{{\^{o}}}1
            {ç}{{\c{c}}}1
            {Ç}{{\c{C}}}1
            {É}{{\'{E}}}1
            {Ê}{{\^{E}}}1
            {À}{{\`{A}}}1
            {Â}{{\^{A}}}1
            {Î}{{\^{I}}}1  
 }

\usepackage{graphicx} % Pour les images 
\usepackage{appendix} % Pour annexes

\usepackage{lmodern}
\usepackage{fancyhdr}

\usepackage[hidelinks]{hyperref} 
\usepackage{url} % Pour écrire des adresses cliquables.
\usepackage[top=2cm, bottom=2cm, left=2cm, right=2cm]{geometry} % Les marges.
\renewcommand{\thesection}{\arabic{section}}

\usepackage{amsmath}
\usepackage{amssymb}
\usepackage{mathrsfs}

\pagestyle{fancy}

\renewcommand{\headrulewidth}{0pt} % Pour enlever la ligne horizontale en haut
\renewcommand{\footrulewidth}{1pt}
\fancyhead[L,R]{}
\fancyfoot[R]{\textbf{page \thepage}} 
\fancyfoot[L]{Rapport : Courbes de Bézier et polices de caractères}
\fancyfoot[C]{}  

\begin{document}

\begin{titlepage}
			\includegraphics[scale=0.25]{Images/Logo_transparent.png} 
			\begin{center}
				\vspace*{6cm}
				{ \huge Rapport Projet PAP \\ \vspace*{2cm}
				\textbf{Courbes de Bézier et polices de caractères} \\
				}
			\vspace*{2cm}
				\begin{center} \large
					\textsc{XU} \textsc{Kevin}\\
					\textsc{LI} \textsc{Ziheng}
				\end{center}
				\begin{minipage}{0.4\textwidth}
				\end{minipage}
				{\large 13 janvier 2019}
			\end{center}
	\end{titlepage}
	
	%Sommaire
	\renewcommand{\contentsname}{Sommaire} 
	{\setlength{\baselineskip}{1.2\baselineskip}
\tableofcontents\par} %RAJOUTER ESPACE ENTRE LIGNE
		
	\addcontentsline{toc}{part}{Préambule} %Ajouter un lien
	\newpage
	\vspace*{3cm} 
	\paragraph{\Huge{Préambule}}

	\paragraph{}
	L'objectif de ce projet est de réalisé des polices de caractères en utilisant des courbes de Bézier. Le programme est implémenté en C++, sans l’aide d’une quelconque bibliothèque extérieure au C++.
	Trois polices de caractères pour les lettres de l’alphabet en majuscule et sans sérif ont été créée : 	
	
	\begin{enumerate}
		\item La première police correspond aux glyphes décrivant le contour du caractère en noir dans un bitmap (rectangle de pixels blancs et noirs, les pixels noirs étant donc le contour du caractère). 
		\item La deuxième police est la première police mais avec l’intérieur des caractères rempli en noir, toujours dans un bitmap. 
		\item La troisième police est la deuxième police mais avec un contour rouge de deux pixels à l’extérieur du caractère. Ce style de police est très utile pour les sous-titrages des films, pour pouvoir toujours lire le sous-titre quelque soit la scène du film. 
 	\end{enumerate}

	Pour fournir les images png de chaque caractère des différentes polices, nous avons utilisé la bibliothèque libpng.
\newline

Nous allons tout d’abord présenter les classes de notre programme. Puis nous évoquerons les objectifs, problèmes rencontrés et les solutions trouvés pour chacune des classes.


	
	\newpage

\section{Les classes (Diagramme UML)}			
\subsection{Diagramme UML}

\section{Point}	
\subsection{Objectifs}
	Tout d'abord, sachant qu'une courbe de Bézier est formée par plusieurs points, il faut avoir une classe définissant un point sur un système de coordonnées dans un plan. Le but de cette classe est alors de définir les point des courbes de Bezier que l'on va tracer. En outre, il faut aussi pouvoir avoir la possibilité de dessiner des points de différentes couleurs.
	
\subsection{Solution}
Une instance de la classe \texttt{Point} a 3 attributs : 
\begin{itemize}
\item La coordonnée $x_$
\item La coordonnée $y_$
\item Un tableau $color_$ contenant les niveaux de Rouge, Vert et de Bleu (allant de 0 à 255). 

Cette classe contient aussi les getters et setters pour ces attributs.

\section{Image}	
% Objectifs		
\subsection{Réalisation}
\subsection{Solution}

\newpage
\section{Courbes de Bézier}	
% Objectifs		
\subsection{Algorithme de de Casteljau}

\texttt{points\_} contient les points de contrôle de la courbe de Bézier et $step\_ = 0.00001$ correspond à la valeur du paramètre lors du calcul du barycentre entre deux points.

\begin{algorithm}
	\caption{\texttt{getCasteljauPoint}}
		\begin{algorithmic}[1]
		\Require $c \in \mathbb{N}$, $index \in \mathbb{N}, t \in \mathbb{R}^{*}$, $points$ A vector which contains the control points of the Bezier Curve
		\Ensure Point
		\Function{getCasteljauPoint}{$c$, $index$, $t$, $points$}
		\If{ c = 0 }
			\State \Return points[index] 
		\EndIf
		\State Set a Point in $P1$ to \texttt{getCasteljauPoint(c-1, index, t, points)} 
		\State Set a Point in $P2$ to \texttt{getCasteljauPoint(c-1, index+1, t, points)} 
		\State Set a Point in $P$ with $x = (1-t) \times (x \; of \; P1) + t \times (x \; of \; P2)$ and $y = (1-t) \times (y \; of \; P1) + t \times (y \; of \; P2)$
		\State \Return $P$
		\EndFunction
		\end{algorithmic}
\end{algorithm}

\begin{algorithm}
	\caption{\texttt{getCurvePoints}}
		\begin{algorithmic}[1]
		\Require $points$ The control points of the Bezier Curve, $step$ The parameter of the barycenter
		\Ensure A vector $Res$ of Points which picture the Bezier Curve
		\Function{getCurvePoints}{$points$, $step$}
		\State Set an empty vector $Res$ of Points
		\State Set $size$ to the size of the vector points\_
		\For{$t = 0$ to $1$ with a step of $step$}
			\State Add to the vector $Res$ the Point : \texttt{getCasteljauPoint(size-1, 0, t)} 
		\EndFor
		\State \Return $Res$
		\EndFunction
		\end{algorithmic}
\end{algorithm}
	
	
\subsection{Solution}

\section{Police 1}	
% Objectifs		
\subsection{Réalisation}
\subsection{Solution}

\newpage
\section{Police 2}	
% Objectifs	
\subsection{Solution}

$img$ est l'image sur laquelle on a déjà dessiné les contours d'une lettre.
 
\begin{algorithm}
	\caption{\texttt{colorInBlack}}
		\begin{algorithmic}[1]
		\Require $x \in \mathbb{N}$, $y \in \mathbb{N}$
		\Ensure Color the inside of a letter in black
		\Function{colorInBlack}{}
		\If{The coordinates $(x, y)$ are out of bounds}
			\State \Return
		\EndIf
		\If{The pixel of coordinates $(x, y)$ is white}
			\State Color the pixel of coordinates $(x, y)$ in black on the Image $img$
			\State \texttt{colorInBlack(x+1, y)} 
			\State \texttt{colorInBlack(x-1, y)} 
			\State \texttt{colorInBlack(x, y+1)} 
			\State \texttt{colorInBlack(x, y-1)} 
		\EndIf	
		\EndFunction
		\end{algorithmic}
\end{algorithm}

\section{Police 3}			
% Objectifs	
\subsection{Solution}

L'image a une taille de 500x500 pixels.
\begin{algorithm}
	\caption{\texttt{addRedContour}}
		\begin{algorithmic}[1]
		\Require An Image $img$ with a letter drawn by the class FontV2
		\Ensure Add a contour of two pixels around the letter
		\Function{addRedContour}{$img$}
		\For{$x \in [\![1;500]\!]$}
			\For{$y \in [\![1;500]\!]$}
				\If{The pixel of coordinates $(x, y)$ is black}
					\If{The pixel of coordinates $(x-1, y)$ is white or red}
						\State Color the pixel of coordinates $(x-1, y)$ in red on the Image $img$
						\State Color the pixel of coordinates $(x-2, y)$ in red on the Image $img$
						\EndIf
					\If{The pixel of coordinates $(x+1, y)$ is white or red}
						\State Color the pixel of coordinates $(x+1, y)$ in red on the Image $img$
						\State Color the pixel of coordinates $(x+2, y)$ in red on the Image $img$
					\EndIf
					\If{The pixel of coordinates $(x, y-1)$ is white or red}
						\State Color the pixel of coordinates $(x, y-1)$ in red on the Image $img$
						\State Color the pixel of coordinates $(x, y-2)$ in red on the Image $img$
					\EndIf
					\If{The pixel of coordinates $(x, y+1)$ is white or red}
						\State Color the pixel of coordinates $(x, y+1)$ in red on the Image $img$
						\State Color the pixel of coordinates $(x, y+2)$ in red on the Image $img$ 
					\EndIf	
				\EndIf	
			\EndFor
		\EndFor
		\EndFunction
		\end{algorithmic}
\end{algorithm}
 

\end{document}
