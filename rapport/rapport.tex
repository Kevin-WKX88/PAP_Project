\documentclass[a4paper, 12pt]{article}

\usepackage[utf8]{inputenc} % POUR MODIFIER : ALLER DANS PREFERENCE TEXSHOP
\usepackage[T1]{fontenc}
\usepackage[francais]{babel}
\usepackage{verbatim}
\usepackage{moreverb}
\usepackage{algorithm, algpseudocode}%Meilleur que listing???

\usepackage{listings} % Pour écrire algo

\usepackage{listingsutf8}

%Paramètrage avec : 
\renewcommand{\lstlistingname}{Interface}
%\lstset{
%basicstyle=\footnotesize,    % taille de la police du code
%numbers=left,   % placer le numéro de chaque ligne à gauche (left) 
%numberstyle=\normalsize,    % taille de la police des numéros
%numbersep=8pt,  % distance entre le code et sa numérotation
%}
\lstset{
language=c++,        % choix du langage de programmation 
  frame=single,
  basicstyle=\small\normalfont\sffamily,    % the size of the fonts that are used for the code
  stepnumber=1,                           % the step between two line-numbers. If it is 1 each line will be numbered
 % numbersep=10pt,                         % how far the line-numbers are from the code
  tabsize=1,                              % tab size in blank spaces
  breaklines=true,                        % sets automatic line breaking
  captionpos=t,                           % sets the caption-position to top
  mathescape=true,
  %stringstyle=\color{white}\ttfamily, % Farbe der String
  showspaces=false,           % Leerzeichen anzeigen ?
  showtabs=false,             % Tabs anzeigen ?
  xleftmargin=17pt,
  framexleftmargin=17pt,
  framexrightmargin=17pt,
  framexbottommargin=5pt,
  framextopmargin=5pt,
  showstringspaces=false,
  inputencoding=utf8,
  extendedchars=true,
  literate=%
            {é}{{\'{e}}}1
            {è}{{\`{e}}}1
            {ê}{{\^{e}}}1
            {ë}{{\¨{e}}}1
            {û}{{\^{u}}}1
            {ù}{{\`{u}}}1
            {â}{{\^{a}}}1
            {à}{{\`{a}}}1
            {î}{{\^{i}}}1
            {ô}{{\^{o}}}1
            {ç}{{\c{c}}}1
            {Ç}{{\c{C}}}1
            {É}{{\'{E}}}1
            {Ê}{{\^{E}}}1
            {À}{{\`{A}}}1
            {Â}{{\^{A}}}1
            {Î}{{\^{I}}}1  
 }

\usepackage{graphicx} % Pour les images 
\usepackage{appendix} % Pour annexes

\usepackage{lmodern}
\usepackage{fancyhdr}

\usepackage[hidelinks]{hyperref} 
\usepackage{url} % Pour écrire des adresses cliquables.
\usepackage[top=2cm, bottom=2cm, left=2cm, right=2cm]{geometry} % Les marges.
\renewcommand{\thesection}{\arabic{section}}

\usepackage{amsmath}
\usepackage{amssymb}
\usepackage{mathrsfs}

\pagestyle{fancy}

\renewcommand{\headrulewidth}{0pt} % Pour enlever la ligne horizontale en haut
\renewcommand{\footrulewidth}{1pt}
\fancyhead[L,R]{}
\fancyfoot[R]{\textbf{page \thepage}} 
\fancyfoot[L]{Rapport : Courbes de Bézier et polices de caractères}
\fancyfoot[C]{}  

\begin{document}

\begin{titlepage}
			\includegraphics[scale=0.25]{Images/Logo_transparent.png} 
			\begin{center}
				\vspace*{6cm}
				{ \huge Rapport Projet PAP \\ \vspace*{2cm}
				\textbf{Courbes de Bézier et polices de caractères} \\
				}
			\vspace*{2cm}
				\begin{center} \large
					\textsc{XU} \textsc{Kevin}\\
					\textsc{LI} \textsc{Ziheng}
				\end{center}
				\begin{minipage}{0.4\textwidth}
				\end{minipage}
				{\large 13 janvier 2019}
			\end{center}
	\end{titlepage}
	
	%Sommaire
	\renewcommand{\contentsname}{Sommaire} 
	{\setlength{\baselineskip}{1.2\baselineskip}
\tableofcontents\par} %RAJOUTER ESPACE ENTRE LIGNE
		
	\addcontentsline{toc}{part}{Préambule} %Ajouter un lien
	\newpage
	\vspace*{3cm} 
	\paragraph{\Huge{Préambule}}

	\paragraph{}
	L'objectif de ce projet est réalisé des polices de caractères en utilisant des courbes de Bézier.\\
\begin{itemize}
\item un actif sans risque : $S_t^0$ à l'instant $t$  
\item un actif risqué (une action) : $S_t$ une variable aléatoire 
\end{itemize}
	
	On va utiliser une fonction $f:  \mathbb{R}_+ \rightarrow \mathbb{R}_+$ tout au long du problème. Cette fonction renvoie le montant d'argent gagné pour un certain montant de l'actif risqué en paramètre.\\
		
	\newpage

\section{Les classes (Diagramme UML)}			
\subsection{Diagramme UML}
\subsection{Question 1}
On a $q_N=\mathbb{Q}(T_1^{(N)}=1+h_N)$ donc $1-q_N=\mathbb{Q}(T_1^{(N)}=1+b_N)$ car $T_1^{(N)}$ ne 

\section{Image}	
% Objectifs		
\subsection{Réalisation}
\subsection{Solution}

\section{Point}	
% Objectifs		
\subsection{Solution}

\section{Courbes de Bézier}	
% Objectifs		
\subsection{Algorithme de de Casteljau}

\texttt{points\_} contient les points de contrôle de la courbe de Bézier et $ratio\_ = 0.00001$ correspond à la valeur du paramètre lors du calcul du barycentre entre deux points.

\begin{algorithm}
	\caption{\texttt{getCasteljauPoint}}
		\begin{algorithmic}[1]
		\Require $c \in \mathbb{N}$, $index \in \mathbb{N}, t \in \mathbb{R}^{*}$
		\Ensure Point
		\Function{getCasteljauPoint}{$c$, $index$, $t$}
		\If{ c = 0 }
			\State \Return points\_[index] 
		\EndIf
		\State Set a Point in P1 to \texttt{getCasteljauPoint(c-1, index, t)} 
		\State Set a Point in P2 to \texttt{getCasteljauPoint(c-1, index+1, t)} 
		\State Set a Point in P with $x = (1-t) \times (x \; of \; P1) + t \times (x \; of \; P2)$ and $y = (1-t) \times (y \; of \; P1) + t \times (y \; of \; P2)$
		\State \Return $P$
		\EndFunction
		\end{algorithmic}
\end{algorithm}

\begin{algorithm}
	\caption{\texttt{getCurvePoints}}
		\begin{algorithmic}[2]
		\Require 
		\Ensure A vector $Res$ of Points
		\Function{getCurvePoints}{}
		\State Set an empty vector $Res$ of Points
		\State Set S to the size of the vector points\_
		\For{$t = 0$ to $1$ with a step of $ratio\_$}
			\State Add to the vector $Res$ the Point : \texttt{getCasteljauPoint(size-1, 0, t)} 
		\EndFor
		\State \Return $Res$
		\EndFunction
		\end{algorithmic}
\end{algorithm}
	
	
\subsection{Solution}

\section{Police 1}	
% Objectifs		
\subsection{Réalisation}
\subsection{Solution}

\section{Police 2}	
% Objectifs	
\subsection{Solution}

\section{Police 3}			
% Objectifs	
\subsection{Solution}
	
\end{document}
